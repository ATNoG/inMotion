\RequirePackage[T1]{fontenc}
\documentclass[conference]{IEEEtran}
\IEEEoverridecommandlockouts

\usepackage{cite}
\usepackage{amsmath,amssymb,amsfonts}
\usepackage{algorithmic}
\usepackage{algorithm}
\usepackage{graphicx}
\usepackage{textcomp}
\usepackage[english]{babel}
\usepackage[dvipsnames]{xcolor}
\usepackage{booktabs}

\def\BibTeX{{\rm B\kern-.05em{\sc i\kern-.025em b}\kern-.08em
T\kern-.1667em\lower.7ex\hbox{E}\kern-.125emX}}

\usepackage[colorlinks=true, linkcolor=blue, urlcolor=BlueViolet, citecolor=blue]{hyperref}

\title{eucnc}
\author{andrepedroribeiro }
\date{January 2026}

\begin{document}
\title{RSSI-Based Passenger Movement Classification for Non-Intrusive Public Transport Monitoring}

\author{\IEEEauthorblockN{Author One\IEEEauthorrefmark{1},
        Author Two\IEEEauthorrefmark{2}, Author Three\IEEEauthorrefmark{3} and
        Author Four\IEEEauthorrefmark{4}}
    \IEEEauthorblockA{Department of Whatever,
        Whichever University\\
        Wherever\\
        Email: \IEEEauthorrefmark{1}author.one@add.on.net,
        \IEEEauthorrefmark{2}author.two@add.on.net,
        \IEEEauthorrefmark{3}author.three@add.on.net,
        \IEEEauthorrefmark{4}author.four@add.on.net}}

\maketitle

\begin{abstract}
Accurate monitoring of passenger flows in public transport is essential for service optimization and network planning, yet traditional counting methods face limitations in coverage, cost, and privacy preservation. This paper presents a novel approach for classifying passenger movements using temporal sequences of WiFi Received Signal Strength Indicator (RSSI) measurements. We introduce a dataset collected in a controlled experimental environment that simulates public transport scenarios, capturing the distinctive signal patterns associated with four fundamental movement classes: boarding the vehicle, alighting from the vehicle, remaining inside, and remaining at the bus stop. By analyzing the temporal evolution of RSSI values over ten-second observation windows, our approach enables non-intrusive distinction between static and transitional states without requiring specialized hardware or compromising passenger anonymity. Experimental evaluation using multiple machine learning classifiers demonstrates the feasibility of RSSI-based movement classification, providing a cost-effective complement to existing automatic passenger counting systems for intelligent transportation applications.
\end{abstract}

\begin{IEEEkeywords}
    Passenger counting, RSSI fingerprinting, WiFi sensing, public transport, machine learning, intelligent transportation systems, urban mobility
\end{IEEEkeywords}

\section{Introduction}\label{sec:introduction}

Accurate passenger flow data is essential for service optimization and network planning, yet traditional counting approaches---manual surveys, ticketing systems, infrared sensors, and \ac{APC} devices---suffer from high costs, incomplete coverage, and privacy concerns \cite{pronello2023evaluating}. The ubiquity of Wi-Fi-enabled devices offers new avenues for non-intrusive sensing: \ac{RSSI} signatures enable movement pattern recognition while preserving anonymity, using only standard networking equipment \cite{nitti2020iabacus, simoncic2023nonintrusive, agualimpia2024rssi}.

This paper proposes a passenger boarding and alighting classification framework based on temporal \ac{RSSI} sequences. By tracking signal evolution over ten-second observation windows, the methodology distinguishes four movement patterns: remaining inside the vehicle, remaining at the stop, boarding, and alighting. Without requiring precise localization or specialized hardware. The principal contributions are:
\begin{enumerate}
  \item An \ac{RSSI}-based movement classification framework exploiting temporal signal evolution;
  \item An experimental dataset comprising approximately 1,360 labelled samples across four classes, collected from four smartphone models spanning three brands;
  \item A comprehensive evaluation of 38 machine learning classifiers with Bayesian hyperparameter optimization;
  \item Feature importance analysis revealing the discriminative role of initial and mid-trajectory \ac{RSSI} measurements;
  \item A privacy-preserving sensing approach that operates without device identification.
\end{enumerate}

The remainder of this paper reviews related work (\autoref{sec:related-work}), describes the experimental setup (\autoref{sec:experimental-setup}), presents exploratory data analysis (\autoref{sec:eda}), presents results (\autoref{sec:results}), and concludes with future directions (\autoref{sec:conclusions}).
\section{Related Work}\label{sec:related-work}

The challenge of accurately monitoring passenger flows in public transport has attracted considerable research attention, resulting in diverse technological approaches ranging from dedicated sensing hardware to opportunistic wireless signal analysis. This section reviews the most relevant contributions across three interconnected domains: automatic passenger counting systems, WiFi-based sensing for mobility applications, and RSSI fingerprinting techniques for localization and classification.

\subsection{Automatic Passenger Counting Systems}

Automatic Passenger Counting (APC) systems have been deployed in public transport networks worldwide to gather ridership data essential for service planning and optimization. Traditional APC technologies primarily rely on infrared sensors, pressure mats, or video-based detection systems installed at vehicle entrances \cite{barabino2014offline}. Pronello and Garzón Ruiz \cite{pronello2023evaluating} conducted a comparative evaluation of commercial video-based APC systems under real-world conditions, revealing that claimed accuracy rates of 98\% often deteriorate significantly in practice, with observed accuracies ranging between 53\% and 74\% for boarding and alighting detection. Their findings underscore the gap between laboratory performance and field deployment, motivating the search for alternative sensing modalities.

Computer vision approaches have advanced considerably with the advent of deep learning techniques. Moreno Rendon et al. \cite{moreno2023passenger} developed a crowd counting system for TransMilenio stations in Bogota using density map estimation, achieving a mean absolute error of approximately one person per frame. Similarly, Wiboonsiriruk et al. \cite{wiboonsiriruk2023efficient} employed the EfficientDet object detection algorithm combined with tracking to count passengers in public transport vehicles, reporting an accuracy of 94\%. More recently, Ono et al. \cite{ono2025realtime} proposed an edge-based solution using Sony's IMX500 AI camera integrated with Raspberry Pi, achieving 100\% accuracy for individual passenger detection while maintaining passenger privacy by processing data locally without transmitting raw video.

Despite these advances, vision-based systems remain constrained by occlusion problems, varying lighting conditions, and mounting requirements that may not be feasible across all vehicle types. Furthermore, the computational demands of real-time video processing can be prohibitive for large-scale deployments, and privacy concerns persist regarding the collection of visual data, even when processed locally.

\subsection{WiFi-Based Sensing for Passenger Detection}

The ubiquity of WiFi-enabled personal devices has motivated research into wireless signal analysis for mobility monitoring. Myrvoll et al. \cite{myrvoll2017counting} pioneered the use of WiFi signatures from mobile devices for public transport passenger counting, demonstrating the feasibility of detecting device presence through probe request analysis. Building upon this foundation, Nitti et al. \cite{nitti2020iabacus} developed iABACUS, a WiFi-based automatic bus passenger counting system that tracks devices throughout their journey without requiring any action from passengers. Their system achieved 100\% detection accuracy in static scenarios and approximately 94\% in dynamic conditions, with errors primarily occurring when consecutive bus stops were closely spaced.

Channel State Information (CSI) has emerged as a particularly promising modality for WiFi sensing, offering richer information about the propagation environment compared to RSSI alone. Wang and Ho \cite{wang2024csi} investigated CSI-based stationary crowd counting using multiple transceiver pairs on double-decker buses, extending the countable range from 11 to 20 passengers with an average accuracy of 90.83\%. The authors emphasized the importance of optimal transceiver topology based on Fresnel zone considerations to maximize sensing performance. Subsequently, Guo et al. \cite{guo2024rssi} proposed an RSSI-assisted CSI-based passenger counting system that leverages multiple WiFi receivers and edge computing, achieving accuracy and F1-scores exceeding 94\% through adaptive feature fusion techniques.

While CSI-based approaches offer superior discriminative capabilities, they require specialized hardware for channel estimation and are computationally more demanding than RSSI-based alternatives. In contrast, RSSI measurements are readily available from standard networking equipment, making them attractive for cost-effective and scalable deployments.

Fabre et al. \cite{fabre2025machine} recently proposed using machine learning algorithms to estimate bus ridership from WiFi sensor data. Their work compared Random Forest, Light Gradient Boosting Machine, and Multilayer Perceptron models for predicting boarding and alighting counts, finding that LGBM provided the most accurate estimates while effectively capturing spatio-temporal variability. This approach addresses the challenge of incomplete WiFi detection rates by learning correction factors from available ground truth data.

\subsection{RSSI Fingerprinting and Classification}

RSSI fingerprinting has been extensively studied for indoor localization, where spatial databases of signal strength measurements are used to infer device positions \cite{zholamanov2025rssi}. Agualimpia-Arriaga et al. \cite{agualimpia2024rssi} presented a comprehensive review of RSSI-based indoor localization using machine learning, categorizing approaches into fingerprinting and trilateration techniques while highlighting the potential of incremental learning for adapting to environmental changes. Jain et al. \cite{jain2021lowcost} demonstrated a low-cost Bluetooth Low Energy (BLE) based indoor localization system achieving 96\% accuracy using Random Forest classification on augmented fingerprint data.

Recent work has extended RSSI analysis beyond static position estimation to trajectory and movement classification. Wang et al. \cite{wang2025learning} proposed a sequence-to-sequence WiFi fingerprinting framework that treats continuously measured RSSI values across multiple timestamps as temporal sequences, enabling more robust localization in dynamic environments. Sarsodia et al. \cite{sarsodia2025rssi} evaluated multiple machine learning models including Support Vector Machines, Random Forest, and K-Nearest Neighbors for RSSI-based indoor localization, with Random Forest achieving the highest accuracy of 96.88\%.

The application of machine learning to RSSI classification in mobility contexts has also gained attention. Servizi et al. \cite{servizi2023truth} addressed the challenge of bus boarding and alighting detection using smartphone-based Bluetooth sensing, introducing variational auto-encoder classifiers and analyzing the impact of human validation errors on model performance. Their work highlighted the complexity of distinguishing transitional states in real-world conditions and the importance of robust feature engineering.

For privacy-preserving crowd monitoring, Simončič et al. \cite{simoncic2023nonintrusive} developed a non-intrusive WiFi probe request detection system that groups similar network management messages using novel clustering and matching procedures. Their de-randomization method achieved over 96\% device detection accuracy when devices were validated separately, demonstrating the viability of anonymous presence monitoring despite MAC address randomization schemes implemented in modern operating systems.

\subsection{Origin-Destination Matrix Estimation}

Accurate characterization of passenger journeys requires not only counting at individual stops but also understanding complete origin-destination patterns. Cerqueira et al. \cite{cerqueira2022inference} proposed a methodology for dynamic OD matrix inference from smart card data, extending traditional approaches with aggregated statistics that highlight network vulnerabilities and multimodal mobility patterns. Tang et al. \cite{tang2024origin} introduced a graph-based deep learning model that distinguishes between direct and transfer trips for OD matrix prediction, addressing the heterogeneous nature of passenger journeys in complex bus networks.

The integration of multiple data sources has proven valuable for comprehensive OD estimation. Dib et al. \cite{dib2023unified} developed a geo-statistical model that combines Automated Fare Collection and APC data to extrapolate occupancy across entire public transport networks, addressing the partial coverage limitations of individual data sources. Similarly, Zhao et al. \cite{zhao2024origin} proposed a multi-modal weighted graph approach for OD matrix estimation that accounts for transfers between different transport modes.

\subsection{Summary}

The reviewed literature demonstrates significant progress in passenger monitoring technologies, yet several gaps remain. Vision-based systems offer high accuracy but raise privacy concerns and face practical deployment challenges. CSI-based WiFi sensing achieves excellent performance but requires specialized equipment. RSSI-based approaches provide a middle ground, combining accessibility with reasonable accuracy, though most existing work focuses on either static counting or position estimation rather than movement classification.

Our work addresses this gap by developing an RSSI-based approach specifically designed to classify passenger movements, distinguishing between boarding, alighting, and stationary states through temporal pattern analysis. This complements existing counting methods by providing richer information about passenger flow dynamics while maintaining the accessibility and privacy benefits of RSSI-based sensing.
\section{Experimental Setup}\label{sec:experimental-setup}

To evaluate the feasibility of RSSI-based passenger movement classification, a series of controlled experiments were conducted in an indoor environment designed to emulate real-world public transport interactions. The experimental setup aimed to reproduce the spatial constraints of a bus and an adjacent bus stop, while enabling reproducible data collection under both isolated and noisy conditions. This section describes the environmental configuration, the data acquisition methodology, and the preprocessing pipeline applied to the collected data.

\subsection{Environmental Configuration}

The experimental environment was divided into two distinct zones, as illustrated in Fig.~\ref{fig:experimental_setup}, representing the interior of a public transport vehicle and the corresponding boarding area and bus stop. Zone~A, designated as the Vehicle Interior, consisted of a closed room used to simulate the interior of a bus. The room provides moderate isolation from external interference, allowing controlled observation of RSSI variations. A WiFi access point (AP) was installed inside the room, positioned adjacent to the doorway, replicating a realistic placement of onboard communication equipment near the vehicle entrance. Zone~B, representing the Bus Stop, was established in the corridor immediately outside the room and was designated as the boarding area. This zone represents the external environment where passengers wait before boarding or after alighting from the vehicle.

The physical separation imposed by the wall and door between the two zones introduces signal attenuation, generating distinctive RSSI patterns when a device transitions between Zone~A and Zone~B. These characteristics are essential for replicating real-world conditions and for enabling the discrimination between static passenger states and transitional movements.

% imagem como placeholder? deveria se fazer outra?
\begin{figure}[!ht]
    \centering
    \includegraphics[width=0.7\linewidth]{images/experimental_setup.png}
    \caption{Experimental environment used to simulate a public transport scenario.}
    \label{fig:experimental_setup}
\end{figure}

\subsection{Data Acquisition Methodology}

Data collection was performed in real time using a custom Python-based extraction script. The script interfaced directly with the WiFi access point via an Ethernet connection, enabling continuous retrieval of network metadata associated with connected devices.

Each experimental trial was conducted over a \textbf{10-second observation window}, during which \textbf{10 consecutive RSSI samples} were recorded at a sampling rate of \textbf{1 Hz}, such that each sample was acquired one second apart. This temporal resolution was selected to capture the dynamic evolution of signal strength as a subject moves within or between zones, over a time interval representative of typical boarding or alighting actions in public transport, while remaining compatible with real-time processing constraints.

To ensure consistent RSSI reporting during data collection, the mobile devices generated periodic low-overhead network traffic (ICMP echo requests) toward the access point. This approach maintained active communication with the AP while introducing negligible additional interference.

For each detected device, three primary attributes were extracted: the \textbf{MAC Address} served as a unique device identifier, used solely for separating different devices during preprocessing, the \textbf{RSSI} (dBm), representing the Received Signal Strength Indicator, served as the primary feature for movement classification and \textbf{Traffic Metadata}, consisting of transmitted and received byte counters, was collected by the AP but not retained for model training.

To account for hardware heterogeneity, four different mobile devices, from three separate manufacturers and different generations, were used throughout the data collection process, introducing variability in antenna characteristics and transmission power. This diversity improves the robustness of the resulting dataset and reduces device-specific bias.

\subsection{Movement Classes and Experimental Scenarios}

Four fundamental movement classes were defined to cover all possible passenger state transitions relative to the vehicle. The first class, Remaining Inside \textbf{(A~$\rightarrow$~A)}, corresponds to static presence or localized movement within the vehicle interior. The second class, Remaining at Stop \textbf{(B~$\rightarrow$~B)}, represents static presence or localized movement within the bus stop area. The third class, Alighting \textbf{(A~$\rightarrow$~B)}, captures the transition from inside the vehicle to the bus stop through the front door. Finally, the fourth class, Boarding \textbf{(B~$\rightarrow$~A)}, describes the transition from the bus stop into the vehicle through the front door.

To evaluate system robustness under varying interference conditions, data collection was performed under two distinct scenarios. The first scenario, Isolated Collection, involved trials conducted with a single active device at a time, minimizing channel contention and external interference. Two of the four devices were used in this scenario, with 20 repetitions per device for each movement class, providing clean baseline RSSI signatures. The second scenario, Noisy or Simultaneous Collection, involved trials conducted with all four devices operating simultaneously, performing two paired different movement classes in parallel. This setup introduces signal interference and collisions, approximating realistic passenger density conditions in public transport environments.

\subsection{Data Preprocessing and Structuring}

The raw data captured by the access point consisted of nested \texttt{Python} dictionaries, where each one-second capture contained network metadata for all connected devices. For each 10-second trial, ten such snapshots were recorded.

The preprocessing pipeline transformed this raw data into a machine-learning-ready dataset through a sequence of four steps. First, Temporal Aggregation was performed, where for each device and trial, the 10 sequential RSSI measurements were aggregated into a single feature vector $\mathbf{R} = [r_1, r_2, \ldots, r_{10}]$, allowing classifiers to exploit temporal trends, slopes, and variance rather than relying solely on instantaneous signal strength. Second, Device Isolation was applied, where data corresponding to each device was isolated using its MAC address, enabling independent trajectory reconstruction even during simultaneous collection scenarios. Third, Labeling was performed by assigning each RSSI sequence the corresponding movement class (\textbf{AA}, \textbf{BB}, \textbf{AB}, or \textbf{BA}) along with a boolean noise label indicating whether the trial was collected in isolation or under simultaneous device activity. Fourth, Feature Filtering was applied, where non-essential attributes, such as transmitted/received byte counts and connection duration, were discarded to reduce dimensionality and prevent overfitting.

\subsection{Final Dataset Structure}

The resulting dataset was exported in a CSV format, where each row represents a complete 10-second trajectory for a single device. Each instance includes the device MAC, 10 RSSI features, the movement class label, and the noise indicator% (\ref{fig:data_sample})
.

% \begin{figure}[!ht]
%     \centering
%     \includegraphics[width=0.9\linewidth]{images/data_sample.png}
%     \caption{Dataset sample}
%     \label{fig:data_sample}
% \end{figure}

This structure enables the analysis of both absolute signal strength and its temporal evolution, which is critical for distinguishing between static presence and transitional passenger movements, forming a solid foundation for supervised learning experiments.
\section{Results}\label{sec:results}

This section presents the experimental results from training and evaluating an extensive set of machine learning classifiers on the \ac{RSSI}-based passenger movement dataset across three experimental scenarios: combined, isolated-only, and noisy-only conditions. The evaluation covers 38 classification algorithms assessed across multiple random seeds to ensure statistical robustness.

\subsection{Comparative Analysis Across Experimental Scenarios}

\autoref{tab:scenario_comparison} presents the performance comparison of top classifiers across the three experimental scenarios, revealing the substantial impact of data collection conditions on classification performance.

\begin{table}[!ht]
    \centering
    \caption{Performance Comparison Across Experimental Scenarios (\ac{MCC})}
    \label{tab:scenario_comparison}
    \resizebox{\columnwidth}{!}{%
        \begin{tabular}{lccc}
            \toprule
            \textbf{Classifier}     & \textbf{Combined}          & \textbf{Isolated-Only}     & \textbf{Noisy-Only}        \\
            \midrule
            KNN (k=5)               & 0.692 $\pm$ 0.029          & \textbf{0.907 $\pm$ 0.061} & 0.704 $\pm$ 0.025          \\
            KNN (k=3)               & 0.690 $\pm$ 0.046          & 0.882 $\pm$ 0.068          & 0.702 $\pm$ 0.043          \\
            LinearSVC               & 0.726 $\pm$ 0.039          & 0.867 $\pm$ 0.060          & 0.716 $\pm$ 0.031          \\
            LogisticRegression (L2) & 0.743 $\pm$ 0.044          & 0.866 $\pm$ 0.028          & 0.731 $\pm$ 0.009          \\
            SVC (Linear)            & 0.750 $\pm$ 0.023          & 0.850 $\pm$ 0.088          & 0.731 $\pm$ 0.037          \\
            StackingEnsemble        & 0.749 $\pm$ 0.020          & 0.851 $\pm$ 0.122          & 0.768 $\pm$ 0.023          \\
            ExtraTrees              & 0.737 $\pm$ 0.013          & 0.836 $\pm$ 0.041          & 0.755 $\pm$ 0.021          \\
            GaussianProcess         & \textbf{0.756 $\pm$ 0.033} & 0.414 $\pm$ 0.052          & 0.755 $\pm$ 0.028          \\
            SVC (RBF)               & 0.755 $\pm$ 0.021          & 0.825 $\pm$ 0.101          & 0.754 $\pm$ 0.016          \\
            CatBoost                & 0.746 $\pm$ 0.017          & 0.782 $\pm$ 0.063          & \textbf{0.770 $\pm$ 0.013} \\
            \bottomrule
        \end{tabular}%
    }
\end{table}

The isolated-only scenario yielded the highest classification performance, with \ac{KNN} (k=5) reaching an \ac{MCC} of 0.907---a 20\% improvement over the combined dataset. Simpler classifiers such as \ac{KNN} showed the most pronounced performance gains in the isolated scenario.

The noisy-only scenario exhibited performance levels comparable to the combined dataset, with CatBoost attaining the highest \ac{MCC} of 0.770. The \ac{GP} classifier maintained robust performance (\ac{MCC}: 0.755) across both combined and noisy-only conditions, yet showed notable degradation in the isolated scenario (\ac{MCC}: 0.414).

\subsection{Per-Class Analysis}

\autoref{tab:class_performance} presents the per-class accuracy, recall, F1-score, and \ac{MCC} for the best classifier (\ac{GP}).

\begin{table}[!ht]
    \centering
    \caption{Per-Class Performance Metrics (\ac{GP})}
    \label{tab:class_performance}
    \resizebox{\columnwidth}{!}{%
        \begin{tabular}{lcccc}
            \toprule
            \textbf{Class}        & \textbf{Accuracy} & \textbf{Recall} & \textbf{F1-Score} & \textbf{MCC}   \\
            \midrule
            AA (Inside)           & 0.838             & 0.750           & 0.791             & 0.785          \\
            BB (Stop)             & 0.838             & 0.897           & 0.878             & 0.785          \\
            BA (Boarding)         & 0.838             & 0.824           & 0.855             & 0.785          \\
            AB (Alighting)        & 0.838             & 0.882           & 0.828             & 0.785          \\
            \midrule
            \textbf{Weighted Avg} & \textbf{0.838}    & \textbf{0.838}  & \textbf{0.838}    & \textbf{0.785} \\
            \bottomrule
        \end{tabular}%
    }
\end{table}

The static state at the bus stop (BB) exhibited the highest recall (89.7\%) and F1-score (0.878). The boarding movement (BA) achieved recall of 82.4\%, F1-score of 0.855, and \ac{MCC} of 0.785. The alighting class (AB) demonstrated high recall (88.2\%) but lower F1-score (0.828). The static state inside the vehicle (AA) presented the lowest recall (75.0\%) and F1-score (0.791). The overall accuracy reached 83.8\% with an \ac{MCC} of 0.785.

\subsection{Hyperparameter Configuration}

While extensive hyperparameter optimization was conducted for nine classifier families using Optuna (detailed in \autoref{sec:appendix}), the \ac{GP} classifier employed default configurations with a \ac{RBF} kernel, which proved highly effective without explicit tuning. The \ac{GP} configuration is presented in \autoref{tab:hyperparams}.

\begin{table}[!ht]
    \centering
    \caption{\ac{GP} Hyperparameters}
    \label{tab:hyperparams}
    \begin{tabular}{ll}
        \toprule
        \textbf{Parameter}   & \textbf{Value}               \\
        \midrule
        Kernel               & $1.0 \times \text{RBF}(1.0)$ \\
        Kernel Length Scale  & Optimized during fitting     \\
        Optimizer            & L-BFGS-B                     \\
        Max Iterations       & 100                          \\
        Multi-class Strategy & One-vs-Rest                  \\
        \bottomrule
    \end{tabular}
\end{table}

The \ac{RBF} kernel automatically learns the optimal length scale parameter during training, adapting to the intrinsic dimensionality of the \ac{RSSI} temporal sequences. Notably, the \ac{GP} classifier achieved top performance without requiring the extensive hyperparameter search applied to other classifiers, suggesting that its probabilistic formulation is particularly well-suited to the \ac{RSSI} feature space.

\subsection{Confusion Matrix Analysis}

\autoref{fig:confusion_matrix} presents the normalized confusion matrix for the \ac{GP} classifier, which achieved the best average performance across experimental seeds.

\begin{figure}[!ht]
    \centering
    \includegraphics[width=0.85\linewidth]{images/confusion_matrix_GaussianProcess.pdf}
    \caption{Confusion matrix for the \ac{GP} classifier, demonstrating strong diagonal dominance with minimal inter-class confusion.}
    \label{fig:confusion_matrix}
\end{figure}

The confusion matrix reveals that the primary source of classification errors occurs between spatially adjacent classes. The AA class (remaining inside) is occasionally misclassified as BA (boarding), and minor confusion exists between AB (alighting) and BB (remaining at stop).

\subsection{Model Stability Analysis}

To evaluate the robustness of classifier rankings across different experimental conditions, \autoref{fig:accuracy_variability} illustrates the accuracy variability for the top classifiers across the three random seeds.

\begin{figure}[!ht]
    \centering
    \includegraphics[width=\linewidth]{images/mcc_variability.pdf}
    \caption{\ac{MCC} variability for top classifiers, in combined dataset, demonstrating consistent ranking stability.}
    \label{fig:accuracy_variability}
\end{figure}

The analysis confirms that top-performing classifiers maintain consistent relative rankings across seeds, with kernel-based methods (\ac{GP}, SVC) and ensemble approaches (Stacking, CatBoost) displaying the lowest variability.

\subsection{Feature Importance}

Analysis of feature importance across interpretable classifiers revealed that the initial \ac{RSSI} measurements (features 1--3) contribute most significantly to classification decisions, as illustrated in \autoref{fig:feature_importance}.

\begin{figure}[!ht]
    \centering
    \includegraphics[width=\linewidth]{images/mean_feature_importance.pdf}
    \caption{Mean feature importance across classifiers, with standard deviation, to the $[0, 1]$ range using min-max normalization}
    \label{fig:feature_importance}
\end{figure}

The first \ac{RSSI} sample exhibits the highest importance (normalized score: 2.29), followed by samples at positions 6 and 2.
\section{Discussion}\label{sec:discussion}

The experimental results demonstrate the feasibility of using temporal RSSI sequences for non-intrusive passenger movement classification in public transport scenarios. This section analyses the key findings across the three experimental scenarios, examines the influence of data collection conditions, and discusses the practical implications for real-world deployment.

\subsection{Classifier Performance Analysis}

The comparative evaluation across experimental scenarios reveals a nuanced picture of classifier suitability for RSSI-based movement classification. On the combined dataset, the Gaussian Process classifier achieved the highest MCC (0.756), attributable to its probabilistic framework and the flexibility of the RBF kernel in modelling non-linear decision boundaries. Unlike parametric models that assume specific functional forms, Gaussian Processes adapt their complexity to the underlying data distribution, which proves advantageous for RSSI patterns that exhibit complex spatial dependencies.

Support Vector Machines with RBF and linear kernels achieved comparable performance on the combined dataset, confirming that kernel-based methods are well-suited for this classification task. The margin-maximization principle of SVMs provides robust generalization, particularly relevant given the overlap between movement classes observed in the t-SNE visualization.

\subsection{Impact of Data Collection Conditions}

The most striking finding emerges from the comparison between experimental scenarios. The isolated-only dataset yielded substantially superior classification performance, with KNN (k=5) achieving an MCC of 0.907—a 31\% relative improvement over its performance on the combined dataset (MCC: 0.692). This dramatic enhancement is attributable to two primary factors:

\textbf{Signal Clarity:} In isolated conditions, RSSI measurements are unaffected by inter-device interference, co-channel contention, or access point load variations. The resulting signal patterns exhibit clearer temporal trajectories with reduced variance within each movement class.

\textbf{Class Separability:} The absence of noise enables more distinct decision boundaries between movement classes. Simpler classifiers such as KNN, which rely on local neighbourhood structure, benefit disproportionately from this increased separability, as evidenced by their top-ranking performance in the isolated scenario.

Conversely, complex models such as Gaussian Process exhibited severe performance degradation in the isolated scenario (MCC: 0.414), despite achieving the best results on the combined dataset. This apparent paradox is explained by the limited sample size ($n = 159$) of the isolated dataset, which is insufficient for the Gaussian Process to reliably estimate its kernel hyperparameters without overfitting.

The noisy-only scenario, representing the most realistic operational conditions, demonstrated performance levels comparable to the combined dataset. CatBoost achieved the highest MCC (0.770) in this scenario, suggesting that gradient boosting methods are particularly robust to signal interference. The consistency between noisy-only and combined results indicates that the combined dataset's performance is predominantly determined by the noisy samples, which constitute 88\% of the total data.

\subsection{Classifier Selection Guidelines}

The experimental results inform practical recommendations for classifier selection based on deployment context:

\textbf{High-interference environments:} For scenarios with multiple simultaneous devices, gradient boosting methods (CatBoost, XGBoost) and ensemble approaches (StackingEnsemble) offer the best balance of accuracy and robustness.

\textbf{Controlled environments:} In settings with minimal device density, simpler classifiers such as KNN or logistic regression can achieve superior performance while offering reduced computational overhead and improved interpretability.

\textbf{General deployment:} For systems that must operate across varying conditions, SVC with RBF kernel provides consistent performance with acceptable variance, making it a reliable default choice.

\subsection{Per-Class Error Analysis}

The confusion patterns reveal insights into the physical characteristics of each movement class. The static state inside the vehicle (AA) exhibited the lowest recall (75.0\%), primarily due to misclassification as boarding (BA). This confusion is attributable to the spatial proximity of both classes to the access point, resulting in similar high-RSSI signatures. Although the temporal dynamics differ (AA maintains relatively stable values while BA shows an increasing trend), this distinction may be subtle in short observation windows.

Conversely, the bus stop class (BB) achieved the highest recall (89.7\%) due to the consistent signal attenuation caused by the physical barrier separating Zone~B from the access point. The transitional classes (AB and BA) benefited from their characteristic monotonic RSSI trends, with alighting (AB) achieving 88.2\% recall.

\subsection{Feature Importance Insights}

The feature importance analysis revealed that initial RSSI measurements (samples 1--3) contribute most significantly to classification decisions, capturing the starting position and providing immediate context for distinguishing static states from transitional movements. The elevated importance of sample 6 (mid-trajectory) indicates that classifiers also rely on signal evolution to confirm movement direction, validating the choice of sequential RSSI measurements over aggregate statistics.

\subsection{Limitations and Considerations}

Several limitations should be acknowledged. First, the controlled experimental environment, while designed to simulate public transport conditions, may not capture all sources of variability present in operational settings, such as passenger density fluctuations, vehicle movement, and diverse access point placements.

Second, the 10-second observation window, while suitable for capturing typical boarding and alighting actions, may be insufficient for detecting slower movements or hesitant passengers. Adaptive window lengths could potentially improve classification accuracy in such cases.

Third, the current approach assumes consistent device behaviour; however, variations in device hardware, operating system power management, and user-initiated WiFi state changes may affect RSSI reporting in practice.

Fourth, the limited sample size of the isolated dataset ($n = 159$) constrains the reliability of performance estimates for complex classifiers in that scenario. Future work should expand isolated data collection to enable more robust comparisons.

\subsection{Practical Implications}

The experimental findings carry significant implications for real-world deployment. The substantial performance improvement observed in isolated conditions (MCC up to 0.907) suggests that signal interference is the primary limiting factor for classification accuracy. Deployment strategies that mitigate interference—such as dedicated frequency channels, directional antennas, or temporal multiplexing—could substantially enhance system performance.

Nevertheless, the achieved classification performance under noisy conditions (MCC $>$ 0.77 with CatBoost) demonstrates that RSSI-based movement classification remains viable as a complementary technology for passenger counting systems even in challenging environments. The consistency of results across the combined and noisy-only scenarios provides confidence that models trained on realistic data will generalize appropriately to operational settings.
\section{Contributions}\label{sec:contributions}

This work puts forward the following contributions:

\begin{enumerate}
    \item \textbf{Novel RSSI-based Movement Classification Framework:} A methodology exploiting temporal RSSI evolution to distinguish static and transitional movement patterns without precise localization.

    \item \textbf{Purpose-Built Experimental Dataset:} Approximately 1,360 labelled samples across four movement classes, collected using four devices under isolated and noisy conditions.

    \item \textbf{Comprehensive Classifier Evaluation:} Comparative analysis of 38 machine learning classifiers with multiple metrics and statistical validation across three random seeds.

    \item \textbf{Feature Importance Analysis:} Identification that initial and mid-trajectory RSSI measurements contribute most to classification accuracy.

    \item \textbf{Privacy-Preserving Approach:} Operation using only aggregate RSSI measurements without device identification or personal data.
\end{enumerate}

Experimental results reaching over 81\% accuracy underscore the practical viability of RSSI-based passenger movement sensing as a cost-effective complement to existing APC technologies.
\section{Conclusions}\label{sec:conclusions}

This paper has presented a new approach for classifying passenger movements in public transport using temporal sequences of WiFi \ac{RSSI} measurements. The proposed methodology enables non-intrusive distinction between four fundamental movement classes using only standard WiFi access point infrastructure.

The \ac{GP} classifier attained the highest performance, with an accuracy of 81.6\%, recall of 81.6\%, F1-score of 81.5\%, and \ac{MCC} of 0.756, validated across multiple random seeds. \acp{SVM} and regularized logistic regression variants yielded comparable results, confirming that both kernel-based and linear methods can effectively exploit the temporal structure of \ac{RSSI} sequences.

Per-class analysis revealed that static states at the bus stop (BB) and transitional movements (AB, BA) are more readily distinguished owing to their characteristic signal patterns, whereas the static state inside the vehicle (AA) posed the greatest classification challenge because of its proximity-based similarity with the boarding class.

The results demonstrate that \ac{RSSI}-based passenger movement classification offers a viable, cost-effective, and privacy-preserving complement to existing \ac{APC} technologies, requiring no specialized hardware beyond standard WiFi infrastructure.

\subsection{Future Work}\label{subsec:future-work}

Several directions for future research emerge from this work. First, validation in operational public transport environments is essential to gauge the impact of real-world factors such as vehicle movement, passenger density variations, and diverse access point configurations. Second, the integration of complementary sensor modalities such as accelerometer data or Bluetooth Low Energy beacons could enhance classification accuracy and robustness. Third, the development of adaptive observation windows that adjust to movement speed could improve detection of hesitant or slower passengers. Fourth, investigation of federated learning approaches would enable model improvement across multiple vehicles while preserving data privacy. Finally, the extension of the methodology to estimate complete \ac{OD} matrices through temporal aggregation of boarding and alighting events represents a natural progression toward comprehensive passenger flow analytics.

\bibliographystyle{IEEEtran}
\bibliography{references}

\end{document}
