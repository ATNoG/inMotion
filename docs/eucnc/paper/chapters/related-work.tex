\section{Related Work}\label{sec:related-work}

\subsection{Automatic Passenger Counting}

Conventional \ac{APC} systems rely on infrared sensors, pressure mats, or video-based detection \cite{barabino2014offline}, yet claimed accuracies of 98\% frequently drop to 53--74\% under real-world conditions \cite{pronello2023evaluating}. Vision-based approaches can reach 94\% accuracy \cite{wiboonsiriruk2023efficient} but remain hampered by occlusion, lighting variability, and privacy concerns.

\subsection{Wi-Fi-Based Sensing and Research Gap}

Wi-Fi-based passenger counting has been demonstrated with up to 100\% detection in static settings \cite{nitti2020iabacus}. While \ac{CSI}-based systems yield richer information with over 94\% accuracy \cite{guo2024rssi}, they require specialized hardware. \ac{RSSI}-based methods remain practical with off-the-shelf equipment; Simon\v{c}i\v{c} et al. \cite{simoncic2023nonintrusive} attained over 96\% accuracy for presence monitoring despite \ac{MAC} randomization challenges.

Despite this progress, machine learning classification of movement patterns from \ac{RSSI} time series has received comparatively little attention. Existing work treats \ac{RSSI} primarily as static fingerprints for indoor localization \cite{agualimpia2024rssi} rather than as temporal sequences encoding movement dynamics. Our work addresses this gap by casting passenger movement detection as a supervised classification problem over temporal \ac{RSSI} evolution, enabling real-time trajectory classification with a single access point at the vehicle door.