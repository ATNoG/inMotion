\section{Related Work}\label{sec:related-work}

\subsection{Automatic Passenger Counting Systems}

Conventional \ac{APC} systems rely on infrared sensors, pressure mats, or video-based detection \cite{barabino2014offline}. Pronello and Garz\'{o}n Ruiz \cite{pronello2023evaluating} observed that claimed 98\% accuracy frequently drops to 53--74\% under real-world conditions. Deep learning approaches can reach up to 94\% accuracy \cite{wiboonsiriruk2023efficient}, though vision-based systems remain hampered by occlusion, lighting variability, and privacy concerns.

\subsection{Wi-Fi-Based Passenger Sensing}

The ubiquity of Wi-Fi-enabled devices has spurred interest in wireless signal analysis for mobility monitoring. Myrvoll et al. \cite{myrvoll2017counting} pioneered probe request analysis for passenger counting, while Nitti et al. \cite{nitti2020iabacus} reported 100\% detection in static settings and 94\% in dynamic ones. \ac{CSI}-based systems yield richer information, with Guo et al. \cite{guo2024rssi} reaching over 94\% accuracy, though they require specialized hardware. \ac{RSSI} remains practical with off-the-shelf equipment. Fabre et al. \cite{fabre2025machine} found Light Gradient Boosting effective for Wi-Fi-based ridership estimation, and Simon\v{c}i\v{c} et al. \cite{simoncic2023nonintrusive} attained over 96\% accuracy despite \ac{MAC} randomization.

\subsection{RSSI Fingerprinting and Movement Classification}

\ac{RSSI} fingerprinting is well-established for indoor localization \cite{agualimpia2024rssi}. Wang et al. \cite{wang2025learning} proposed treating \ac{RSSI} as temporal sequences, aligning with our methodology. Servizi et al. \cite{servizi2023truth} addressed boarding detection using Bluetooth sensing, while Cerqueira et al. \cite{cerqueira2022inference} demonstrated the importance of complete journey patterns for \ac{OD} matrix inference.

\subsection{Research Gap}

Despite progress in passenger counting and wireless sensing, machine learning classification of movement patterns from \ac{RSSI} time series has received comparatively little attention. Existing approaches tend to focus on aggregate counting rather than fine-grained movement classification. Our work addresses this gap by casting passenger movement detection as a supervised classification problem using temporal \ac{RSSI} evolution, thereby enabling real-time trajectory classification with a single access point at the vehicle door and keeping infrastructure requirements to a minimum.