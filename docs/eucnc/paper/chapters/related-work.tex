\section{Related Work}\label{sec:related-work}

The challenge of accurately monitoring passenger flows in public transport has attracted considerable research attention, resulting in diverse technological approaches ranging from dedicated sensing hardware to opportunistic wireless signal analysis. This section reviews the most relevant contributions across three interconnected domains: automatic passenger counting systems, WiFi-based sensing for mobility applications, and RSSI fingerprinting techniques for localization and classification.

\subsection{Automatic Passenger Counting Systems}

Automatic Passenger Counting (APC) systems have been deployed in public transport networks worldwide to gather ridership data essential for service planning and optimization. Traditional APC technologies primarily rely on infrared sensors, pressure mats, or video-based detection systems installed at vehicle entrances \cite{barabino2014offline}. Pronello and Garzón Ruiz \cite{pronello2023evaluating} conducted a comparative evaluation of commercial video-based APC systems under real-world conditions, revealing that claimed accuracy rates of 98\% often deteriorate significantly in practice, with observed accuracies ranging between 53\% and 74\% for boarding and alighting detection. Their findings underscore the gap between laboratory performance and field deployment, motivating the search for alternative sensing modalities.

Computer vision approaches have advanced considerably with the advent of deep learning techniques. Moreno Rendon et al. \cite{moreno2023passenger} developed a crowd counting system for TransMilenio stations in Bogota using density map estimation, achieving a mean absolute error of approximately one person per frame. Similarly, Wiboonsiriruk et al. \cite{wiboonsiriruk2023efficient} employed the EfficientDet object detection algorithm combined with tracking to count passengers in public transport vehicles, reporting an accuracy of 94\%. More recently, Ono et al. \cite{ono2025realtime} proposed an edge-based solution using Sony's IMX500 AI camera integrated with Raspberry Pi, achieving 100\% accuracy for individual passenger detection while maintaining passenger privacy by processing data locally without transmitting raw video.

Despite these advances, vision-based systems remain constrained by occlusion problems, varying lighting conditions, and mounting requirements that may not be feasible across all vehicle types. Furthermore, the computational demands of real-time video processing can be prohibitive for large-scale deployments, and privacy concerns persist regarding the collection of visual data, even when processed locally.

\subsection{WiFi-Based Sensing for Passenger Detection}

The ubiquity of WiFi-enabled personal devices has motivated research into wireless signal analysis for mobility monitoring. Myrvoll et al. \cite{myrvoll2017counting} pioneered the use of WiFi signatures from mobile devices for public transport passenger counting, demonstrating the feasibility of detecting device presence through probe request analysis. Building upon this foundation, Nitti et al. \cite{nitti2020iabacus} developed iABACUS, a WiFi-based automatic bus passenger counting system that tracks devices throughout their journey without requiring any action from passengers. Their system achieved 100\% detection accuracy in static scenarios and approximately 94\% in dynamic conditions, with errors primarily occurring when consecutive bus stops were closely spaced.

Channel State Information (CSI) has emerged as a particularly promising modality for WiFi sensing, offering richer information about the propagation environment compared to RSSI alone. Wang and Ho \cite{wang2024csi} investigated CSI-based stationary crowd counting using multiple transceiver pairs on double-decker buses, extending the countable range from 11 to 20 passengers with an average accuracy of 90.83\%. The authors emphasized the importance of optimal transceiver topology based on Fresnel zone considerations to maximize sensing performance. Subsequently, Guo et al. \cite{guo2024rssi} proposed an RSSI-assisted CSI-based passenger counting system that leverages multiple WiFi receivers and edge computing, achieving accuracy and F1-scores exceeding 94\% through adaptive feature fusion techniques.

While CSI-based approaches offer superior discriminative capabilities, they require specialized hardware for channel estimation and are computationally more demanding than RSSI-based alternatives. In contrast, RSSI measurements are readily available from standard networking equipment, making them attractive for cost-effective and scalable deployments.

Fabre et al. \cite{fabre2025machine} recently proposed using machine learning algorithms to estimate bus ridership from WiFi sensor data. Their work compared Random Forest, Light Gradient Boosting Machine, and Multilayer Perceptron models for predicting boarding and alighting counts, finding that LGBM provided the most accurate estimates while effectively capturing spatio-temporal variability. This approach addresses the challenge of incomplete WiFi detection rates by learning correction factors from available ground truth data.

\subsection{RSSI Fingerprinting and Classification}

RSSI fingerprinting has been extensively studied for indoor localization, where spatial databases of signal strength measurements are used to infer device positions \cite{zholamanov2025rssi}. Agualimpia-Arriaga et al. \cite{agualimpia2024rssi} presented a comprehensive review of RSSI-based indoor localization using machine learning, categorizing approaches into fingerprinting and trilateration techniques while highlighting the potential of incremental learning for adapting to environmental changes. Jain et al. \cite{jain2021lowcost} demonstrated a low-cost Bluetooth Low Energy (BLE) based indoor localization system achieving 96\% accuracy using Random Forest classification on augmented fingerprint data.

Recent work has extended RSSI analysis beyond static position estimation to trajectory and movement classification. Wang et al. \cite{wang2025learning} proposed a sequence-to-sequence WiFi fingerprinting framework that treats continuously measured RSSI values across multiple timestamps as temporal sequences, enabling more robust localization in dynamic environments. Sarsodia et al. \cite{sarsodia2025rssi} evaluated multiple machine learning models including Support Vector Machines, Random Forest, and K-Nearest Neighbors for RSSI-based indoor localization, with Random Forest achieving the highest accuracy of 96.88\%.

The application of machine learning to RSSI classification in mobility contexts has also gained attention. Servizi et al. \cite{servizi2023truth} addressed the challenge of bus boarding and alighting detection using smartphone-based Bluetooth sensing, introducing variational auto-encoder classifiers and analyzing the impact of human validation errors on model performance. Their work highlighted the complexity of distinguishing transitional states in real-world conditions and the importance of robust feature engineering.

For privacy-preserving crowd monitoring, Simončič et al. \cite{simoncic2023nonintrusive} developed a non-intrusive WiFi probe request detection system that groups similar network management messages using novel clustering and matching procedures. Their de-randomization method achieved over 96\% device detection accuracy when devices were validated separately, demonstrating the viability of anonymous presence monitoring despite MAC address randomization schemes implemented in modern operating systems.

\subsection{Origin-Destination Matrix Estimation}

Accurate characterization of passenger journeys requires not only counting at individual stops but also understanding complete origin-destination patterns. Cerqueira et al. \cite{cerqueira2022inference} proposed a methodology for dynamic OD matrix inference from smart card data, extending traditional approaches with aggregated statistics that highlight network vulnerabilities and multimodal mobility patterns. Tang et al. \cite{tang2024origin} introduced a graph-based deep learning model that distinguishes between direct and transfer trips for OD matrix prediction, addressing the heterogeneous nature of passenger journeys in complex bus networks.

The integration of multiple data sources has proven valuable for comprehensive OD estimation. Dib et al. \cite{dib2023unified} developed a geo-statistical model that combines Automated Fare Collection and APC data to extrapolate occupancy across entire public transport networks, addressing the partial coverage limitations of individual data sources. Similarly, Zhao et al. \cite{zhao2024origin} proposed a multi-modal weighted graph approach for OD matrix estimation that accounts for transfers between different transport modes.

\subsection{Summary}

The reviewed literature demonstrates significant progress in passenger monitoring technologies, yet several gaps remain. Vision-based systems offer high accuracy but raise privacy concerns and face practical deployment challenges. CSI-based WiFi sensing achieves excellent performance but requires specialized equipment. RSSI-based approaches provide a middle ground, combining accessibility with reasonable accuracy, though most existing work focuses on either static counting or position estimation rather than movement classification.

Our work addresses this gap by developing an RSSI-based approach specifically designed to classify passenger movements, distinguishing between boarding, alighting, and stationary states through temporal pattern analysis. This complements existing counting methods by providing richer information about passenger flow dynamics while maintaining the accessibility and privacy benefits of RSSI-based sensing.