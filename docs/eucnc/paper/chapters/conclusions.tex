\section{Conclusions}\label{sec:conclusions}

This paper has presented a new approach for classifying passenger movements in public transport using temporal sequences of WiFi \ac{RSSI} measurements. The proposed methodology enables non-intrusive distinction between four fundamental movement classes using only standard WiFi access point infrastructure.

The \ac{GP} classifier attained the highest performance, with an accuracy of 81.6\%, recall of 81.6\%, F1-score of 81.5\%, and \ac{MCC} of 0.756, validated across multiple random seeds. \acp{SVM} and regularized logistic regression variants yielded comparable results, confirming that both kernel-based and linear methods can effectively exploit the temporal structure of \ac{RSSI} sequences.

Per-class analysis revealed that static states at the bus stop (BB) and transitional movements (AB, BA) are more readily distinguished owing to their characteristic signal patterns, whereas the static state inside the vehicle (AA) posed the greatest classification challenge because of its proximity-based similarity with the boarding class.

The results demonstrate that \ac{RSSI}-based passenger movement classification offers a viable, cost-effective, and privacy-preserving complement to existing \ac{APC} technologies, requiring no specialized hardware beyond standard WiFi infrastructure.

\subsection{Future Work}\label{subsec:future-work}

Several directions for future research emerge from this work. First, validation in operational public transport environments is essential to gauge the impact of real-world factors such as vehicle movement, passenger density variations, and diverse access point configurations. Second, the integration of complementary sensor modalities such as accelerometer data or Bluetooth Low Energy beacons could enhance classification accuracy and robustness. Third, the development of adaptive observation windows that adjust to movement speed could improve detection of hesitant or slower passengers. Fourth, investigation of federated learning approaches would enable model improvement across multiple vehicles while preserving data privacy. Finally, the extension of the methodology to estimate complete \ac{OD} matrices through temporal aggregation of boarding and alighting events represents a natural progression toward comprehensive passenger flow analytics.