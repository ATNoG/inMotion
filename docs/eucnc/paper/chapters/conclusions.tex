\section{Conclusions}\label{sec:conclusions}

This paper has presented a new approach for classifying passenger movements in public transport using temporal sequences of Wi-Fi \ac{RSSI} measurements. The proposed methodology enables non-intrusive distinction between four fundamental movement classes using only standard Wi-Fi access point infrastructure.

The \ac{GP} classifier attained the highest performance on the combined dataset, with an \ac{MCC} of 0.756, validated across multiple random seeds. \acp{SVM} and regularized logistic regression variants yielded comparable results, confirming that both kernel-based and linear methods can effectively exploit the temporal structure of \ac{RSSI} sequences.

Per-class analysis revealed that static states at the bus stop and transitional movements are more readily distinguished owing to their characteristic signal patterns, whereas the static state inside the vehicle posed the greatest classification challenge because of its proximity-based similarity with the boarding class.

The results demonstrate that \ac{RSSI}-based passenger movement classification offers a viable, cost-effective, and privacy-preserving complement to existing \ac{APC} technologies, requiring no specialized hardware beyond standard Wi-Fi infrastructure.

Future work includes validation in operational public transport environments, integration of complementary sensor modalities, development of adaptive observation windows, and extension of the methodology to estimate complete \ac{OD} matrices through temporal aggregation of boarding and alighting events.