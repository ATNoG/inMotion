\section{Exploratory Data Analysis}\label{sec:eda}

Ahead of classifier training, an exploratory data analysis was carried out to gauge the discriminative potential of temporal \ac{RSSI} signatures and to characterize signal behaviour across different passenger movement classes.

\subsection{Dataset Composition}

The final dataset is roughly balanced, comprising around 340 samples per movement class. For each class, 40 samples were gathered under isolated conditions, while the remaining samples were obtained during simultaneous device activity, thereby introducing controlled signal interference. This balance ensures that the exploratory analysis and subsequent results reflect both ideal and realistic operating conditions.

\subsection{Temporal RSSI Characteristics}

The temporal evolution of \ac{RSSI} values serves as the primary discriminative feature between movement classes. \autoref{fig:temporal_rssi} depicts the mean \ac{RSSI} trajectory over the 10-second observation window for each class.

\begin{figure}[!ht]
    \centering
    \includegraphics[width=\linewidth]{images/rssi_mean_trajectory_per_class.pdf}
    \caption{Temporal evolution of \ac{RSSI} values over the 10-second observation window for each movement class.}
    \label{fig:temporal_rssi}
\end{figure}

Static states (\textbf{AA and BB}) exhibit relatively stable signal levels over time, albeit with distinct average magnitudes due to their spatial separation from the access point. In contrast, transitional movements display clear monotonic trends. The boarding class (\textbf{B~$\rightarrow$~A}) shows a consistent increase in \ac{RSSI} as the devices move toward the access point, while the alighting class (\textbf{A~$\rightarrow$~B}) presents a pronounced decrease as physical obstructions attenuate the signal.

These contrasting temporal patterns offer a compelling rationale for leveraging \ac{RSSI} sequences in movement classification.
