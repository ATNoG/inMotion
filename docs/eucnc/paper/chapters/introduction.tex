\section{Introduction}\label{sec:introduction}

Urban mobility poses significant challenges as over 75\% of EU citizens reside in cities, with transportation contributing approximately 24\% of greenhouse gas emissions \cite{eurostat2023, eea2023transport}. Effective public transport management requires accurate passenger flow data, yet traditional counting approaches---manual surveys, ticketing systems, infrared sensors, and automated passenger counting (APC) devices---present limitations including high costs, incomplete coverage, and privacy concerns \cite{barabino2014offline, pronello2023evaluating}. Additionally, conventional ticketing often fails to capture complete passenger journeys due to absent exit validation \cite{cerqueira2022inference}.

The proliferation of WiFi-enabled devices offers new opportunities for non-intrusive passenger sensing. These devices generate Received Signal Strength Indicator (RSSI) signatures that enable movement pattern recognition while preserving anonymity \cite{nitti2020iabacus, simoncic2023nonintrusive}. While Channel State Information (CSI) approaches achieve high accuracy, they require specialized hardware and intensive computation \cite{guo2024rssi}. RSSI-based methods provide a practical alternative using standard networking equipment \cite{agualimpia2024rssi}.

This paper presents a novel approach to passenger boarding and alighting classification using temporal RSSI sequences. By analyzing signal evolution over ten-second observation windows, our methodology distinguishes four movement patterns: remaining inside the vehicle, remaining at the stop, boarding, and alighting---without requiring precise localization.

The main contributions include: (1) a novel RSSI-based movement classification framework exploiting temporal signal evolution; (2) an experimental dataset of approximately 1,360 labelled samples across four classes; (3) comprehensive evaluation of 38 machine learning classifiers; (4) feature importance analysis; and (5) a privacy-preserving sensing approach operating without device identification.

The remainder of this paper reviews related work (\autoref{sec:related-work}), describes the experimental setup (\autoref{sec:experimental-setup}), presents exploratory data analysis (\autoref{sec:eda}), presents results (\autoref{sec:results}) and discussion (\autoref{sec:discussion}), and concludes with future directions (\autoref{sec:conclusions}).