\section{Introduction}\label{sec:introduction}

Accurate passenger flow data is essential for public transport optimization, yet traditional counting approaches---manual surveys, ticketing systems, infrared sensors, and \ac{APC} devices---suffer from high costs, incomplete coverage, and privacy concerns \cite{pronello2023evaluating}. The ubiquity of Wi-Fi-enabled devices offers new avenues for non-intrusive sensing: \ac{RSSI} signatures enable movement pattern recognition while preserving anonymity, using only standard networking equipment \cite{nitti2020iabacus, simoncic2023nonintrusive, agualimpia2024rssi}.

This paper proposes a passenger boarding and alighting classification framework based on temporal \ac{RSSI} sequences. By tracking signal evolution over ten-second observation windows, the methodology distinguishes four movement patterns: remaining inside the vehicle, remaining at the stop, boarding, and alighting. Without requiring precise localization or specialized hardware. The principal contributions are:
\begin{enumerate}
    \item An \ac{RSSI}-based movement classification framework exploiting temporal signal evolution;
    \item An experimental dataset comprising approximately 1,360 labelled samples across four classes, collected from four smartphone models spanning three brands;
    \item A comprehensive evaluation of 38 machine learning classifiers with Bayesian hyperparameter optimization;
    \item Feature importance analysis revealing the discriminative role of initial and mid-trajectory \ac{RSSI} measurements;
    \item A privacy-preserving sensing approach that operates without device identification.
\end{enumerate}