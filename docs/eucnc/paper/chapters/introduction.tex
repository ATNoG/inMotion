\section{Introduction}\label{sec:introduction}

Urban mobility remains a pressing concern, as over 75\% of EU citizens reside in cities and transportation accounts for roughly 24\% of greenhouse gas emissions \cite{eurostat2023, eea2023transport}. Managing public transport effectively hinges on accurate passenger flow data, yet traditional counting approaches---manual surveys, ticketing systems, infrared sensors, and automated passenger counting (APC) devices---come with drawbacks such as high costs, incomplete coverage, and privacy concerns \cite{barabino2014offline, pronello2023evaluating}. Moreover, conventional ticketing often fails to capture complete passenger journeys when exit validation is absent \cite{cerqueira2022inference}.

The widespread adoption of WiFi-enabled devices opens up new avenues for non-intrusive passenger sensing. Such devices generate Received Signal Strength Indicator (RSSI) signatures that allow movement pattern recognition while preserving anonymity \cite{nitti2020iabacus, simoncic2023nonintrusive}. Although Channel State Information (CSI) approaches can achieve high accuracy, they demand specialized hardware and intensive computation \cite{guo2024rssi}. RSSI-based methods, by contrast, offer a practical alternative using standard networking equipment \cite{agualimpia2024rssi}.

This paper puts forward a new approach to passenger boarding and alighting classification using temporal RSSI sequences. By tracking signal evolution over ten-second observation windows, our methodology distinguishes four movement patterns: remaining inside the vehicle, remaining at the stop, boarding, and alighting---all without requiring precise localization.

The principal contributions of this work are: (1) an RSSI-based movement classification framework that exploits temporal signal evolution; (2) an experimental dataset comprising approximately 1,360 labelled samples across four classes; (3) a thorough evaluation of 38 machine learning classifiers; (4) feature importance analysis; and (5) a privacy-preserving sensing approach that operates without device identification.

The remainder of this paper reviews related work (\autoref{sec:related-work}), describes the experimental setup (\autoref{sec:experimental-setup}), presents exploratory data analysis (\autoref{sec:eda}), presents results (\autoref{sec:results}) and discussion (\autoref{sec:discussion}), and concludes with future directions (\autoref{sec:conclusions}).