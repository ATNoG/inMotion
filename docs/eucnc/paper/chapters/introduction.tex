\section{Introduction}\label{sec:introduction}

Urban mobility represents one of the most pressing challenges facing modern cities, as more than 75\% of European Union citizens currently reside in urban areas \cite{eurostat2023}. Inefficient transportation systems contribute approximately 24\% of greenhouse gas emissions, underscoring the urgent need for intelligent and sustainable mobility solutions \cite{eea2023transport}. Public transport networks play a pivotal role in addressing these challenges; however, their effective management requires accurate and continuous data regarding passenger flows across the network.

Traditional approaches to passenger counting and origin-destination (OD) matrix estimation rely on manual surveys, ticketing systems, or dedicated hardware such as infrared sensors and automated passenger counting (APC) devices \cite{barabino2014offline}. While these methods have proven useful, they present significant limitations including high deployment costs, incomplete spatial coverage, privacy concerns associated with video-based systems, and the inability to provide continuous, non-intrusive monitoring of passenger movements \cite{pronello2023evaluating}. Furthermore, conventional ticketing systems often fail to capture the complete passenger journey, as exit validation is frequently absent in many public transport networks \cite{cerqueira2022inference}.

The proliferation of personal wireless devices, particularly smartphones, has opened new avenues for non-intrusive passenger detection and tracking. WiFi-enabled devices continuously emit probe requests and maintain connections with nearby access points, generating Received Signal Strength Indicator (RSSI) signatures that can be leveraged for localization and movement pattern recognition \cite{nitti2020iabacus}. Unlike camera-based systems, WiFi sensing approaches inherently preserve passenger anonymity while providing valuable insights into mobility patterns \cite{simoncic2023nonintrusive}.

Recent advances in Channel State Information (CSI) extraction have further enhanced the capabilities of WiFi-based passenger counting systems, achieving remarkable accuracy rates exceeding 90\% in controlled environments \cite{guo2024rssi}. However, CSI-based approaches require specialized hardware and are computationally intensive, limiting their practical deployment in resource-constrained settings. In contrast, RSSI-based methods offer a more accessible alternative, requiring only standard networking equipment while still providing meaningful signal fingerprints for classification tasks \cite{agualimpia2024rssi}.

This paper presents a novel approach to passenger boarding and alighting classification using temporal sequences of RSSI measurements. The proposed methodology enables the distinction between four fundamental movement patterns: remaining inside the vehicle, remaining at the bus stop, boarding the vehicle, and alighting from the vehicle. By analyzing the temporal evolution of signal strength over a short observation window, our approach captures the distinctive signatures associated with each movement pattern without requiring precise localization or continuous tracking of individual devices.

The main contributions of this work are threefold. First, we introduce a dataset specifically designed for RSSI-based passenger movement classification, collected in a controlled experimental environment that simulates real-world public transport scenarios. Second, we present a comprehensive analysis of temporal RSSI patterns across different movement classes, demonstrating the feasibility of distinguishing between static and transitional states. Third, we evaluate multiple machine learning classifiers for this task, providing insights into the most effective approaches for real-time passenger flow estimation.

The remainder of this paper is organized as follows. Section \ref{sec:related-work} reviews related work in passenger counting, WiFi-based sensing, and RSSI fingerprinting. Section \ref{sec:experimental-setup} describes the experimental setup and data collection methodology. Section \ref{sec:results} presents the classification results, followed by a discussion in Section \ref{sec:discussion}. Finally, Section \ref{sec:conclusions} concludes the paper and outlines directions for future work.