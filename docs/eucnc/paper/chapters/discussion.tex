\section{Discussion}\label{sec:discussion}

The aforementioned experimental results confirm the feasibility of using temporal \ac{RSSI} sequences for non-intrusive passenger movement classification in public transport scenarios. This section analyses the key findings across the three experimental scenarios, examines how data collection conditions influence performance, and discusses the practical implications for real-world deployment.

\subsection{Classifier Performance Analysis}

The comparative evaluation across experimental scenarios paints a nuanced picture of classifier suitability for \ac{RSSI}-based movement classification. On the combined dataset, the \ac{GP} classifier attained the highest \ac{MCC} (0.756), exhibiting strong discriminative capability for the temporal \ac{RSSI} patterns. This performance stems from its probabilistic framework and the flexibility of the \ac{RBF} kernel in modelling non-linear decision boundaries. Unlike parametric models that assume specific functional forms, \acp{GP} adapt their complexity to the underlying data distribution, an advantage for \ac{RSSI} patterns that exhibit complex spatial dependencies.

The flexibility of the \ac{RBF} kernel enables the \ac{GP} to model complex decision boundaries without requiring extensive manual hyperparameter tuning, making it particularly suitable for \ac{RSSI}-based classification where signal patterns exhibit non-linear spatial dependencies. \acp{SVM} with \ac{RBF} and linear kernels achieved comparable performance on the combined dataset, confirming that kernel-based methods are well-suited for this classification task. The margin-maximization principle of \acp{SVM} provides robust generalization, particularly relevant given the overlap between movement classes observed in the \ac{t-SNE} visualization.

\subsection{Impact of Data Collection Conditions}

Perhaps the most striking finding emerges from the comparison between experimental scenarios. The isolated-only dataset yielded markedly superior classification performance, with \ac{KNN} (k=5) reaching an \ac{MCC} of 0.907---a 31\% relative improvement over its performance on the combined dataset (\ac{MCC}: 0.692). This pronounced increase can be traced to the absence of inter-device signal interference during data collection, resulting in cleaner \ac{RSSI} patterns with more distinct class separations. Two primary factors account for this enhancement:

\textbf{Signal Clarity:} In isolated conditions, \ac{RSSI} measurements are unaffected by inter-device interference, co-channel contention, or access point load variations. The resulting signal patterns exhibit clearer temporal trajectories with reduced variance within each movement class.

\textbf{Class Separability:} The absence of noise enables more distinct decision boundaries between movement classes. Simpler classifiers such as \ac{KNN}, which rely on local neighbourhood structure, benefit disproportionately from this increased separability, suggesting that the underlying class boundaries are well-defined when noise is absent. This explains the pronounced performance gains observed for \ac{KNN} in the isolated scenario.

Conversely, complex models such as the \ac{GP} suffered marked performance degradation in the isolated scenario (\ac{MCC}: 0.414), despite attaining the best results on the combined dataset. This apparent paradox is explained by the limited sample size ($n = 160$) of the isolated dataset, which proves insufficient for the \ac{GP} to reliably estimate its kernel hyperparameters without overfitting.

The noisy-only scenario, representing the most realistic operational conditions, showed performance levels on par with the combined dataset. CatBoost attained the highest \ac{MCC} (0.770) in this scenario, suggesting that gradient boosting methods are particularly robust to signal interference. The consistency between noisy-only and combined results indicates that the combined dataset's performance is largely driven by the noisy samples, which make up 88\% of the total data.

\subsection{Model Stability Considerations}

The observed stability of classifier rankings across experimental seeds has notable implications for deployment. Kernel-based methods (\ac{GP}, SVC) and ensemble approaches (Stacking, CatBoost) showed the lowest variability, a desirable property for deployment scenarios where model retraining may occur with different data partitions. This consistency lends confidence that selected classifiers will maintain their relative performance across varying operational conditions.

\subsection{Classifier Selection Guidelines}

The experimental results inform practical recommendations for classifier selection based on deployment context:

\textbf{High-interference environments:} For scenarios with multiple simultaneous devices, gradient boosting methods (CatBoost, XGBoost) and ensemble approaches (StackingEnsemble) offer the best balance of accuracy and robustness.

\textbf{Controlled environments:} In settings with minimal device density, simpler classifiers such as KNN or logistic regression can achieve superior performance while offering reduced computational overhead and improved interpretability.

\textbf{General deployment:} For systems that must operate across varying conditions, SVC with \ac{RBF} kernel provides consistent performance with acceptable variance, making it a reliable default choice.

\subsection{Per-Class Error Analysis}

The confusion patterns reveal insights into the physical characteristics of each movement class. The static state inside the vehicle (AA) exhibited the lowest recall (75.0\%), primarily due to misclassification as boarding (BA). This confusion is attributable to the spatial proximity of both classes to the access point, resulting in similar high-\ac{RSSI} signatures that reflect the similarity in \ac{RSSI} magnitude when devices are positioned near the access point. Although the temporal dynamics differ (AA maintains relatively stable values while BA shows an increasing trend), this distinction may be subtle in short observation windows.

Conversely, the bus stop class (BB) achieved the highest recall (89.7\%), attributable to the consistent low \ac{RSSI} values observed when devices remain outside the vehicle. The physical barrier separating Zone~B from the access point provides natural signal attenuation that enables clear class discrimination. The transitional classes (AB and BA) benefited from their characteristic monotonic \ac{RSSI} trends, with alighting (AB) achieving 88.2\% recall. However, the lower F1-score (0.828) for the alighting class indicates some false positives from the AA class. The boarding movement (BA) achieved strong results (F1-score: 0.855), benefiting from the distinctive increasing \ac{RSSI} pattern as devices approach the access point.

The overall accuracy of 83.8\% and \ac{MCC} of 0.785 confirm robust multi-class discrimination across all movement categories, demonstrating that \ac{RSSI}-based temporal patterns provide sufficient information for reliable passenger movement classification.

\subsection{Confusion Matrix Interpretation}

The observed confusion between spatially adjacent classes provides important insights for system design. The AA--BA confusion reflects the similarity in \ac{RSSI} magnitude when devices are positioned near the access point, suggesting that additional features or longer observation windows might improve discrimination between static and transitional states. Similarly, the AB--BB confusion arises because both classes share lower \ac{RSSI} values characteristic of the exterior zone, though the transitional nature of AB provides some discriminative information through temporal patterns.

\subsection{Feature Importance Insights}

The feature importance analysis revealed that initial \ac{RSSI} measurements (features 1--3) contribute most significantly to classification decisions, capturing the starting position and providing immediate context for distinguishing static states from transitional movements. This finding aligns with the temporal dynamics of passenger movements, where early signal readings capture the initial position before any state transition occurs.

The elevated importance of feature 6 (mid-trajectory) indicates that classifiers also rely on signal evolution to confirm movement direction, validating the choice of sequential \ac{RSSI} measurements over aggregate statistics. The observed pattern suggests that classifiers leverage both the starting signal strength and mid-trajectory measurements to infer movement direction, while later features provide confirmatory information about the final position. This temporal dependency structure supports the use of sequence-based classification approaches for \ac{RSSI}-based passenger detection.

The cross-classifier variability in feature importance, quantified by the standard deviation across normalized importance values, provides additional insight into classifier behaviour. Feature~1 exhibits the highest mean importance (0.998) with the lowest relative variability (standard deviation of 0.009), indicating universal agreement among classifiers regarding the discriminative value of the initial \ac{RSSI} measurement. Conversely, mid-trajectory features (4, 5, 7, and~9) exhibit higher relative variability, with coefficient of variation values exceeding 0.7. This disparity reflects fundamental differences in how classifiers exploit temporal information: tree-based ensemble methods tend to favour early features that establish initial position, whereas kernel-based methods such as \acp{SVM} and \acp{GP} distribute importance more evenly across the temporal sequence to capture signal dynamics. Feature~6 occupies an intermediate position, with moderate importance (0.450) but elevated variability (standard deviation of 0.193), suggesting that its utility for classification is model-dependent and potentially influenced by the specific decision boundaries each classifier learns.

\subsection{Limitations and Considerations}

Several limitations should be acknowledged. First, the controlled experimental environment, while designed to simulate public transport conditions, may not capture all sources of variability present in operational settings, such as passenger density fluctuations, vehicle movement, and diverse access point placements.

Second, the 10-second observation window, while suitable for capturing typical boarding and alighting actions, may be insufficient for detecting slower movements or hesitant passengers. Adaptive window lengths could potentially improve classification accuracy in such cases.

Third, the current approach assumes consistent device behaviour; however, variations in device hardware, operating system power management, and user-initiated WiFi state changes may affect \ac{RSSI} reporting in practice.

Fourth, the limited sample size of the isolated dataset ($n = 160$) constrains the reliability of performance estimates for complex classifiers in that scenario. Future work should expand isolated data collection to enable more robust comparisons.

\subsection{Practical Implications}

The experimental findings hold meaningful implications for real-world deployment. The notable performance improvement observed in isolated conditions (\ac{MCC} up to 0.907) suggests that signal interference is the chief limiting factor for classification accuracy. Deployment strategies that mitigate interference---such as dedicated frequency channels, directional antennas, or temporal multiplexing---could markedly enhance system performance.

Nevertheless, the classification performance achieved under noisy conditions (\ac{MCC} $>$ 0.77 with CatBoost) shows that \ac{RSSI}-based movement classification remains viable as a complementary technology for passenger counting systems, even in challenging environments. The consistency of results across the combined and noisy-only scenarios lends confidence that models trained on realistic data will generalize well to operational settings.